\documentclass[a4paper,11pt]{article}
	\usepackage{beppe_package} %questo pacchetto dovrebbe contenere tutto quello che ci serve
							   %ci sono diversi comandi rinominati, aprite il file per vedere quali sono i comandi veloci
	
	%questi comandi servono per classificare le domande
	\renewcommand{\a}{(a) }
	\renewcommand{\b}{(b) }
	\renewcommand{\c}{(c) }
	
	\title{Risposte per l'orale di Fisica 3}
	\author{Giuseppe Bogna, Edoardo Centamori, Alessandro Foligno, Elia Pizzati, Paolo Tognini}
	
	
\begin{document}
	\maketitle
	\begin{abstract}
		\noindent In questo documento vogliamo raccogliere le risposte alle possibili domande\footnote{La lista delle domande si trova su \url{http://www2.ing.unipi.it/~a008137/fis_3_checklist.pdf}} di Fisica 3 per l'anno accademico 2017-2018. Alle domande contrassegnate con (a) si richiede una risposta pronta e sicura, le domande e gli esercizi contrassegnati con (b) potrebbe non avere una risposta immediata, le domande contrassegnate con (c) sono argomenti a scelta. Se presenti, indichiamo anche riferimenti specifici agli argomenti delle singole domande.
	\end{abstract}
\section{Prerequisiti}
	\begin{enumerate}
		\item \a prima domanda
	\end{enumerate}
\section{Indagine della materia tramite collisioni e decadimenti di particelle}
	\begin{enumerate}
		\item 
	\end{enumerate}
\section{Elettromagnetismo classico e acceleratori di particelle}
	\begin{enumerate}
		\item 
	\end{enumerate}
\section{Interazione radiazione-materia}
	\begin{enumerate}
		\item 
	\end{enumerate}
\section{Domande a scelta (bisogna sceglierne 4 e indicarle all'inizio dell'orale)}
	\begin{enumerate}
		\item Prova
	\end{enumerate}
\end{document}
